\documentclass{article}
\usepackage{amsmath}
\usepackage{amsfonts}
\usepackage[english]{babel}
\usepackage[utf8]{inputenc}
\usepackage{amsthm}
\usepackage{graphicx}

\newcommand{\norm}[1]{\left\lVert#1\right\rVert}
\newtheorem{theorem}{Theorem}
\newtheorem{prop}{Proposition}

\begin{document}

\title{Linear Algebra Hw1}
\author{Toby Harvey}
\maketitle

\noindent(1a)

\begin{gather*}
  \left(\frac{-1 + \sqrt{3}i}{2}\right)^3 = \left(-\frac{1}{2} + \frac{\sqrt{3}i}{2}\right)\left(-\frac{1}{2} + \frac{\sqrt{3}i}{2}\right)\left(-\frac{1}{2} + \frac{\sqrt{3}i}{2}\right) = \left(\frac{1}{4} - \frac{3}{4} - \frac{\sqrt{3}i}{2}\right)\left(-\frac{1}{2} + \frac{\sqrt{3}i}{2}\right)\\
  =  \left(-\frac{1}{2} - \frac{\sqrt{3}i}{2}\right)\left(-\frac{1}{2} + \frac{\sqrt{3}i}{2}\right) = \frac{1}{4} + \frac{3}{4} = 1
\end{gather*}

\vspace{3mm}

\noindent(2a)

\vspace{3mm}

\noindent Let $\alpha, \beta \in \mathbb{C}$ so that $\alpha = a + bi$, $\beta = c+di$. If $\alpha\beta = 1$,
then by the definition of multiplication in $\mathbb{C}$ we have:


$$(ac - bd) + (ad + bc)i = 1 + 0i \implies ac - bd = 1 \text{ and } ad + bc = 0$$

\noindent This is a linear system of two equations in $c$ and $d$. Eliminating the first term in each equation we get:

$$(-abc + b^2d = -b) + (abc + a^2d = 0) \implies b^2d + a^2d = -b \implies d = - \frac{b}{b^2 + a^2}$$

\noindent pluggin this back to $ad + bc = 0$ we get:

$$a\left(\frac{-b}{b^2 + a^2}\right) + bc = 0 \implies c = \frac{a}{b^2 + a^2}$$

\noindent Therefore the multplicative identity in $\mathbb{C}$ is $\frac{a}{b^2 + a^2} - \frac{b}{b^2 + a^2}i$

\newpage

\noindent(11a)

\vspace{3mm}

\noindent By way of contradiction, assume there exists $\lambda \in \mathbb{C}, \lambda = a +bi$ so that:

\begin{gather*}
  \lambda(2 - 3i) = 12 - 5i\\
  \lambda(5 + 4i) = 7 + 22i\\
  \lambda(-6 + 7i) = -32 - 9i\\
\end{gather*}

\noindent expanding this out:

\begin{gather*}
  2a + 3b + (-3a + 2b)i = 12 - 5i\\
  5a - 4b + (4a + 5b)i= 7 + 22i\\
  -6a - 7b + (7a - 6b)i = -32 - 9i\\
\end{gather*}

\noindent or:

\begin{gather*}
  2a + 3b = 12\\
  -3a + 2b = -5\\
   5a - 4b = 7\\
  4a + 5b = 22\\
  -6a - 7b = -32\\
  7a - 6b = -9\\
\end{gather*}

This is a system of 6 equations with 2 unknowns so it is very over determined. We can see it has no solution if we find the a solution to only the first two equations:

$$\left(\frac{3}{2}(2a + 3b) = \frac{3}{2}(12)\right) + (-3a + 2b = -5) \implies \frac{13}{2}b = 13 \implies b = 2 $$

so that $-3a + 4 = -5 \implies a = 3$

Plugging these into the sixth equation we get:

$$7(3) - 6(2) = -9 \implies 9 = -9$$

Which is a contradiction.

\newpage

\noindent(16a)

\begin{gather*}
  (a+b)x =\\
  ((a+b)x_1,(a+b)x_2,...,(a+b)x_n) =\\
  (ax_1 + bx_1, ax_2 + bx_2,...,ax_n + bx_n) =\\
  (ax_1, ax_2,...ax_n) + bx_1,bx_2,...bx_n) =\\
  a (x_1, x_2,...,x_n) + b(x_1, x_2,...,x_n) =
  ax+bx
\end{gather*}

\vspace{3mm}

\noindent(4b)

\vspace{3mm}

\noindent There are no elements in the empty set so the there is no existence of the additive identity.

\vspace{3mm}

\noindent(5b)

\vspace{3mm}

We need to show that the condition $0\vec{v} = \vec{0}$ implies that there exists $\vec{w} = -\vec{v}$ such that $\vec{v} + (-\vec{v}) = 0$:

$$\vec{0} = 0\vec{v} = (1 + (-1))\vec{v} = 1\vec{v} + (-1)\vec{v} = \vec{v} + (-\vec{v})$$

Where in the last equality we have let $-1\vec{v} = (-\vec{v}) =  \vec{w}$.

\vspace{3mm}

\noindent(1c)

\vspace{3mm}

\noindent(a) First the additive identity is in this subspace because $0 + 2(0) = 3(0) \implies 0 = 0$.

\vspace{1mm}
\noindent Addition is closed: Let $W$ denote the subspace in question. Let $x = (x_1, x_2, x_3) \in W$, and $ y = (y_1, y_2, y_3) \in W$. We need to show $x+y$ is still in the subspace:

$$(x_1 + y_1) + 2(x_2 + y_2) + 3(x_3 + y_3) = (x_1 + 2x_2 + 3x_3) + (y_1 + 2y_1 + 3x_3) = 0 + 0 = 0$$

\noindent Multiplication is closed: Let $a\in\mathbb{F}$. We must show that $ax \in W$:

$$ax_1 + 2ax_2 + 3ax_3 = a(x_1 + 2x_2 + 3x_3) = a(0) = 0$$

\noindent So this is a subspace.

\vspace{6mm}

\noindent(b) This is not a subspace of $\mathbb{F}^3$ because it does not contain the additive identity of $\mathbb{F}^3$, $\vec{0}$.

\vspace{3mm}

\noindent(c) This is not a subspace. Example of it not being closed under addition:

$$(1,0,0) , (0, 1, 1) \in W \qquad (1, 0, 0) + (0, 1, 1) = (1, 1, 1)$$

$$1(1)(1) = 1 \implies (1,1,1) \notin W$$


\newpage

\noindent(d) $W$ does contain the addative identity since $0 = 5(0) \implies 0 = 0$.

Addition is closed:
$$(x_1 + y_1) - 5(x_2 + y_2) = (x_1 - 5x_2) + (y_1 - 5y_2) = 0 + 0 = 0$$
Multiplication is closed:
$$ax_1 - 5ax_3 = a(x_1 - 5x_3) = a(0) = 0$$



\end{document}




















